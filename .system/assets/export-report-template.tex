\documentclass{article}
\usepackage[utf8]{inputenc}
\usepackage[top=1cm,bottom=1cm,left=1.8cm,right=1.8cm]{geometry}
\usepackage{eurosym}
\usepackage{colortbl}
\usepackage{graphicx}
\usepackage[french]{babel}
\usepackage{tabularx}
\pagestyle{empty}
\setlength{\parindent}{0pt} % Empêche les indentations de première ligne
\begin{document}

% --- EN-TÊTE ---
\begin{minipage}[t]{.3\linewidth}

    \vspace{0pt}
    % Afficher le logo, s'il est renseigner dans les données
    \BLOCK{ if logo_path }
    \includegraphics[width=\linewidth, keepaspectratio]{\VAR{logo_path}}\\
    \BLOCK{ endif }

    \vspace{0.2cm}

    % Afficher les informations de l'association
    \textbf{\textcolor[RGB]{255,87,51}{\VAR{association_name}}}\\
    \VAR{association_adress_1},\\
    \VAR{association_adress_2}\\
    \VAR{association_email}\\
\end{minipage}
\hfill
\begin{minipage}[t]{.55\linewidth}
    \hrulefill
    \begin{center}
        \Large{\textbf{Note de frais \no \VAR{ER_number}} : \VAR{mandate}-\VAR{ER_number}}

        \vspace{0.2cm}
        \normalsize{Faite le \VAR{date}}
    \end{center}
    \hrulefill

    \vspace{0.5cm}

    % Afficher les informations du bénéficiaire
    \begin{flushright}
        \textbf{\textcolor[RGB]{255,87,51}{Bénéficiaire}}\\
        \VAR{beneficiary_name}\\
        \VAR{beneficiary_adress_1},\\
        \VAR{beneficiary_adress_2}\\
        \BLOCK{ if benficiary_phone }
        \VAR{beneficiary_phone}\\
        \BLOCK{ endif }
    \end{flushright}
\end{minipage}

\vspace{1cm}

% --- TITRE PRINCIPAL ---
\begin{center}
    \large{\textbf{Note de frais}}\\
    \hrulefill
\end{center}

% --- TABLEAU DES DÉPENSES ---
\begin{center}
% Redéfinition de la colonne X dans TabularX
\renewcommand{\tabularxcolumn}[1]{>{\small}m{#1}}
\renewcommand{\arraystretch}{1.5} % Aère les lignes du tableau

% X = Colonne extensible (prend toute la place restante et revient à la ligne auto)
% r = aligné à droite (convnetionnel pour les prix)
% c = centré (conventionnel pour la quantité)
\begin{tabularx}{\linewidth}{|c|X|r|r|}
    \hline
    \textbf{\textcolor[RGB]{255,87,51}{Qté}} &
    \textbf{\textcolor[RGB]{255,87,51}{Référence / Description}} &
    \textbf{\textcolor[RGB]{255,87,51}{Prix Unitaire}} &
    \textbf{\textcolor[RGB]{255,87,51}{Prix Total}} \\
    \hline

    % Boucle sur les items
    \BLOCK{ for item in items }
    \VAR{item.quantity} &
    \VAR{item.reference} &
    \VAR{item.unit_price} \euro &
    \VAR{item.total_price} \euro \\
    \hline
    \BLOCK{ endfor }

    % Ligne du TOTAL
    \multicolumn{3}{|r|}{\textbf{TOTAL :}} &
    \textbf{\VAR{final_price} \euro} \\
    \hline

\end{tabularx}
\end{center}

\vspace{0.5cm}


\textbf{Pièces jointes :} Justificatifs


\vspace{1cm}

% --- SIGNATURES ---
\begin{minipage}[t]{0.45\linewidth}
    \textbf{\textcolor[RGB]{255,87,51}{Signature du bénéficiaire}}\\
    \VAR{beneficiary_name}\\
    Fait à \hspace{2.5cm} le \\

    % Espace pour signature manuelle si besoin
    \vspace{2cm}
\end{minipage}
\hfill
\begin{minipage}[t]{0.45\linewidth}
    \textbf{\textcolor[RGB]{255,87,51}{Signature de la trésorerie}}\\
    \VAR{treasurer}\\
    Fait à \VAR{signature_location} le \VAR{date} \\

    \vspace{0.2cm}

    \BLOCK{ if signature_path }
    % Image responsive
    \includegraphics[width=5cm, keepaspectratio]{\VAR{signature_path}}
    \BLOCK{ endif }
\end{minipage}

\vspace{1cm}

\underline{Demande de remboursement :}
\VAR{refund_mod} \VAR{beneficiary_iban}\\

\newpage

% --- JUSTIFICATIFS (PAGE SUIVANTE) ---
\begin{center}
\large{\textbf{Justificatifs}}\\
\hrulefill
\end{center}

\vspace{0.5cm}

\centering
\BLOCK{ for file in receipt_files }
    % Deux images par lignes
    \begin{minipage}{0.45\linewidth}
        \centering
        \includegraphics[width=\linewidth, keepaspectratio]{\VAR{file}}
        \vspace{0.5cm}
    \end{minipage}
    \hfill % Espace entre les deux images sur la même ligne
\BLOCK{ endfor }

\end{document}
