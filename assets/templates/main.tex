\documentclass{article}
\usepackage[utf8]{inputenc}
\usepackage[top=1cm,bottom=1cm,left=1.8cm,right=1.8cm]{geometry}
\usepackage{eurosym}
\usepackage{colortbl}
\usepackage{graphicx}
\usepackage[frenchb]{babel}
\pagestyle{empty}

\begin{document}

\begin{figure}[!ht]
   \begin{minipage}[c]{.4\linewidth}
        % logo de l'association
        \BLOCK{ if logo_path }
        \includegraphics[width = 7cm]{\VAR{logo_path}}\\
        \BLOCK{ endif }
        % Nom de l'association
        \textbf{\textcolor[RGB]{255,87,51}{\VAR{association_name}}}\\
        % Adresse de l'association
        \VAR{association_adress_1},\\
        \VAR{association_adress_2}\\
        % Email de l'association
        \VAR{association_email}\\

   \end{minipage} \hfill
   \begin{minipage}[l]{.55\linewidth}
        \hrulefill
        \begin{center}
        \begin{Large}
         % Numéro de note de frais  : année du mandat + numéro de la note de frais
        \textbf{Note de frais \no \VAR{ERC_number}} : \VAR{mandate}-\VAR{ERC_number}\\
        \end{Large}
        % Date du jour
        Faite le \VAR{date}\\
        \end{center}
        \hrulefill\\
        \newline
        \begin{flushright}
        \textbf{\textcolor[RGB]{255,87,51}{Bénéficiaire\\}}
        % Nom et prénom du bénéficiaire
        \VAR{beneficiary_name}\\
        % Adresse du bénéficiaire
        \VAR{beneficiary_adress_1},\\
        \VAR{beneficiary_adress_2}\\
        % Coordonnées du bénéficiaire
        \BLOCK{ if benficiary_phone }
        \VAR{beneficiary_phone}\\
        \BLOCK{ endif }


\vspace{0,2cm}

        \end{flushright}
   \end{minipage}
\end{figure}

\begin{center}
\large{\textbf{Note de frais}}\\
\hrulefill
\end{center}
\begin{center}
\begin{large}
\begin{tabular}{>{\centering\arraybackslash} b{1.8cm} >{\centering\arraybackslash} b{7.3cm} >{\centering\arraybackslash} b{3.5cm} >{\centering\arraybackslash} b{3.5cm}}
    \textbf{\textcolor[RGB]{255,87,51}{Quantité}} & \textbf{\textcolor[RGB]{255,87,51}{Référence}} & \textbf{\textcolor[RGB]{255,87,51}{Prix unitaire}} &  \textbf{\textcolor[RGB]{255,87,51}{Prix total}} \\
\end{tabular}

{\renewcommand{\arraystretch}{1.6}
\begin{tabular}{|>{\centering\arraybackslash} p{1.8cm}|>{\centering\arraybackslash} p{7cm}|>{\centering\arraybackslash} p{3.5cm}|>{\centering\arraybackslash} p{3.5cm}|}
  \hline
   % Boucle sur les items
   \BLOCK{ for item in items }
   \VAR{item.quantity} & \VAR{item.reference} & \VAR{item.unit_price} \euro & \VAR{item.total_price} \euro \\
   \BLOCK{ endfor }

  \hline


\end{tabular}}
\end{large}
\end{center}
\begin{flushright}
    \begin{large}
    \begin{tabular}{r l}
            % Ajout du total
            \textbf{TOTAL :} \textbf & \VAR{final_price} \euro\\
    \end{tabular}
    \end{large}
\end{flushright}

\vspace{0,5cm}

% Liste des pèces jointes sur les pages suivantes
\BLOCK{ if attachment_list }
Pièces jointes : \VAR{attachment_list}
\BLOCK{ endif }

\vspace{0,5cm}

\begin{minipage}{0.5\linewidth}
    \textbf{\textcolor[RGB]{255,87,51}{Signature du bénéficiaire}}\\
    % Nom du bénéficiaire
    \VAR{beneficiary_name}\\
    % Lieu + date
    Fait à \hspace{2.5cm} le \\
    % Signature
\end{minipage}
\begin{minipage}{0.5\linewidth}
    \textbf{\textcolor[RGB]{255,87,51}{Signature de la trésorerie}}\\
    % Nom du trésorier/de la trésorière
    \VAR{treasurer}\\
    % Lieu + date
    Fait à \VAR{lieu_signature} le \VAR{date} \\
    % Signature
    \BLOCK{ if signature_path }
    \includegraphics[width = 7cm]{\VAR{signature_path}}
    \BLOCK{ endif }
\end{minipage}

\vspace{0,5cm}

\underline{Demande de remboursement par :}\\
% IBAN du bénéficiaire

- \VAR{refund_mod} \VAR{beneficiary_iban}\\

\newpage

% Preuves de paiments (justificatifs)
\begin{figure}[h!]
    \centering % Centrer l'ensemble du bloc
    \BLOCK{ for file in receipt_files }
    \begin{minipage}{0.4\linewidth}\centering
        \includegraphics[width = \linewidth]{\VAR{file}}
    \end{minipage}
    ~
    \BLOCK{ endfor }
\end{figure}



\end{document}
